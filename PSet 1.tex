\documentclass{article}
\usepackage{graphicx}
\usepackage{amsfonts}
\usepackage{amsmath}
\usepackage{amssymb}
\usepackage{amsthm}

\title{PSet 1}
\author{Aly Rajwani}
\date{September 2024}

\begin{document}

\maketitle

\begin{enumerate}
    \item \textbf{Exercise 1.3.9}
        \begin{enumerate}
            \item By definition of sup $B$, $\forall \epsilon > 0,$ there is an element $b \in B$ s.t. $b > \text{sup }B - \epsilon$. Since $\text{sup }B > \text{sup }A, \text{sup }B - \text{sup }A > 0$, so let $\epsilon = \text{sup }B - \text{sup }A$, which gives $\exists b \in B$ s.t. $b > \text{sup }B - \epsilon = \text{sup }B - (\text{sup }B - \text{sup }A) = \text{sup }A$. 

            Since $b > \text{sup }A$, $b$ is an upper bound for $A$.

            \item 
            Let $A = [0, 1], B = (0, 1)$. Then $\text{sup }A = \text{sup }B = 1$, but $\nexists b \in B$ that is an upper bound for $A$, since $1 \in A$. 
        \end{enumerate}

    \item \textbf{Exercise 1.4.8}
    \begin{enumerate}
        \item Let $A = (0, 1) \cap \mathbf{Q}$ and $B = (0, 1) \cap \mathbf{I}$.

        $A \cap B = \emptyset$, since $A \subset \mathbf{Q}$ and $B \subset \mathbf{I}$.

        $\text{sup }A = \text{sup B} = 1$, and $1 \notin A$ and $1 \notin B$

        \item Let $J_n = \left( -\frac{1}{n}, \frac{1}{n}\right)$. Then, $\bigcap_{n=1}^\infty J_n = \{0\}$ since $\forall n \in \mathbf{N}, -\frac{1}{n} < 0 < \frac{1}{n}$, and $J_{n+1} \subset J_n$ since $\left(-\frac{1}{n+1}, \frac{1}{n+1}\right) \subset \left(-\frac{1}{n}, \frac{1}{n}\right)$. No other element can exist in the intersection since the Archimedean Property states that $\forall \epsilon > 0, \exists n \in \mathbf{N}$ s.t. $\frac{1}{n} < \epsilon$.

        \item Let $L_n = [n, \infty)$. Then $\bigcap_{n=1}^\infty = \emptyset$ since, if you suppose $x$ is in this intersection, choose $n = \lfloor x \rfloor + 1$, so $x \notin L_n$, and $L_{n+1} \subset L_n$ since $[n+1, \infty) \subset [n, \infty)$

        \item This request is impossible. 
        
        Let $J_n = \bigcap_{k=1}^n I_k$. Then, $J_{n+1} = I_{n+1} \cap J_n \subset J_n$, so these interval are nested. Since $\bigcap_{k=1}^n I_k \neq \emptyset$, each $J_n$ is non-empty, and since each $I_n$ is closed, the intersection will also be closed. We can now apply the nested interval property: $\bigcap_{n=1}^\infty I_n = \bigcap_{n=1}^\infty J_n \neq \emptyset$, proving the request was impossible. 
    \end{enumerate}

    \item \textbf{Exercise 1.5.10}
        \begin{enumerate}
            \item We prove this by contradiction. 

            Suppose that $\forall a \in (0, 1), C \cap [a, 1]$ is countable. Consider the union $\bigcup_{n=1}^\infty C \cap \left[\frac{1}{n}, 1\right] = C \cap (0, 1]$. We have this equality since the Archimedean Property states that $\forall \epsilon > 0,\exists n \in \mathbf{N}$ s.t. $\frac{1}{n} < \epsilon$. Since this is the union of countably many countable sets, the union is countable. So, $C \cap (0, 1]$ is countable, which implies $C \cap [0, 1] = C$ is countable, which contradicts the original assertion. Thus, there exists an $a \in (0, 1)$ s.t. $C \cap [a, 1]$ is uncountable. 

            \item We will make a similar argument. Note that since $\alpha$ is the supremum of $A$, $C \cap [\alpha + \epsilon, 1]$ for $\epsilon > 0$ is countable. Also note that if $\alpha = 1$, then we have $C \cap [1, 1]$, which is clearly countable. So, we can assume $\alpha < 1$.

            Consider the union $\bigcup_{n=1}^\infty C \cap [\alpha + \frac{1}{n}, 1] = C \cap (\alpha, 1]$. Since this union is the intersection of countably many countable sets, it is countable. So, $C \cap (\alpha, 1]$ is countable, which implies $C \cap [\alpha, 1]$ is countable. 

            \item It does not, as demonstrated by the following example. 

            Let $C = \{\frac{1}{n}\}_{n \in \mathbf{N}}$. Consider some $a \in (0, 1)$. We seek to show that $C \cap [a, 1]$ is finite, which we do by partitioning $C$ as follows: $C = \left\{\frac{1}{n}\right\}_{n=1}^{\lfloor \frac{1}{a}\rfloor + 1} \cup \left\{\frac{1}{n}\right\}_{n> \lfloor \frac{1}{a} \rfloor + 1}$. Please excuse the sloppy/hard-to-follow notation, but essentially, the first set consists of all elements of $C$ that are greater than $a$, and the second set consists of all other elements. Notably, this first set contains only finitely many elements by its definition. So,
            \begin{align*}
                C \cap [a, 1] &= \left(\left\{\frac{1}{n}\right\}_{n=1}^{\lfloor \frac{1}{a}\rfloor + 1} \cup \left\{\frac{1}{n}\right\}_{n> \lfloor \frac{1}{a} \rfloor + 1}\right) \cap [a, 1] \\
                &= \left\{\frac{1}{n}\right\}_{n=1}^{\lfloor \frac{1}{a}\rfloor + 1} \cap [a, 1]
            \end{align*}

            and this is finite since the left set is finite. 
            
        \end{enumerate}

    \item \textbf{Exercise 1.6.10}
        \begin{enumerate}
            \item Since the domain of each of these functions is only 2 elements, we can represent each of these functions as an ordered pair $(a, b)$, where $a = f(0)$ and $b = f(1)$. The image of these functions is $\mathbf{N}$, and so we can represent this set as $\left\{(a, b) : a, b \in \mathbf{N} \right\}$. This set is equivalent to $\mathbf{N}^2$, which is countable. 

            \item Let $A = \{f: \mathbb{N} \rightarrow \{0, 1\}\}$. Assume $A$ is countable and enumerate the functions it contains as $\{f_n\}_{n\in \mathbb{N}}$. Consider a new function $g: \mathbb{N} \rightarrow \{0, 1\}$ defined as $g(n) = 
            \begin{cases}
                0 \text{ if } f_n(n) = 1 \\
                1 \text{ if } f_n(n) = 0
            \end{cases}$. 

            By its construction, $g$ differs in at least one mapping when compared to each $f$, and so $g \notin A$, but $g$ is a function from $\mathbb{N}$ to $\{0, 1\}$. Thus, enumerating the functions in $A$ leads to a contradiction, and $A$ is uncountable. 

            \item Partition $\mathbb{N}$ as $\left\{\{1, 2\}, \{3, 4\}, \{5, 6\}, ... \right\}$. 
            
            Let $\mathcal{A} = \{ \text{sets containing exactly one element from each region of } \mathbf{N}\}$. As an example, $\{2n: n \in \mathbb{N}\}$ is an element of $\mathcal{A}$. Consider any two elements of $\mathcal{A}$. Either they contain identical elements, in which case they are the same set, or at some region in the partition they contain different elements, in which case they cannot be subsets of each other. Thus, $\mathcal{A}$ is an antichain. 
            

            Assume $\mathcal{A}$ is countable and enumerate the subsets it contains as $\{\mathcal{A}_n\}_{n \in \mathbb{N}}$. Consider a new subset $A$ defined by $\begin{cases}
                2n - 1\in A \text{ if } 2n \in \mathcal{A}_n \\
                2n \in A \text{ if } 2n -1 \in \mathcal{A}_n
            \end{cases}$

            $A$ differs from each $\mathcal{A}_n$ by at least one element, and since we began by partitioning $\mathbb{N}$, this means $A$ is not a subset of any $\mathcal{A}_n$, yet $A \in \mathcal{P}(\mathbb{N})$. Thus enumerating elements of $\mathcal{A}$ leads to a contradiction, and it is uncountable, so $\mathcal{P}(\mathbb{N})$ contains an uncountable anitchain. 
        \end{enumerate}

    \item \textbf{Exercise 2.2.7}
        \begin{enumerate}
            \item Frequently. For any $n$, if the $n$th term of the sequence is 1, then the $n+1$th term is $-1$, and so it is never eventually in the set $\{1\}$.

            \item Eventually implies frequently, but frequently does not imply eventually. For the latter case, see the previous example. For the former, note that if beyond a certain $N$, each term in the sequence is in $(a_n)$, then for any $n$, take $k = \text{max}(N, n)$, so $k \geq n$ and $a_k \in A$, which is the definition of being frequently in a set. 

            \item A sequence $(a_n)$ converges to $a$ if, given any $\epsilon$, $(a_n)$ is eventually in $V_\epsilon(a)$.

            \item Consider $\{1, 2, 1, 2, 1, 2, \dots\}$. This set is not eventually in $(1.9, 2.1)$ since $\forall n$, if the $n$th term is in $(1.9, 2.1)$, the $n+1$th term is not. However, for any $N$, there are only finitely many terms $a_n$ with $n < N$. So, there are infinitely many terms $a_n$ with $n \geq N$ and $a_n = 2$. So, it is frequently in $(1.9, 2.1)$.
        \end{enumerate}
        
    \item \textbf{Exercise 2.3.5}

        This is a biconditional so we prove 2 statements. First, that $(z_n)$ converges if $(x_n)$ and $(y_n)$ are convergent with $\text{lim }x_n = \text{lim }y_n$. Second, that $(x_n)$ and $(y_n)$ are convergent with $\text{lim }x_n = \text{lim }y_n$ if $(z_n)$ converges. 

        \textit{Forwards: }

        If $(x_n)$ converges to $l$, then $\forall \epsilon > 0, \exists N_1 \in \mathbf{N}$ s.t. $\forall n \geq N_1, |x_n - l| < \epsilon$. Similarly, $\forall \epsilon > 0, \exists N_2 \in \mathbf{N}$ s.t. $\forall n \geq N_2, |y_n - l| < \epsilon$.

        Choose $N = \text{max}(N_1, N_2)$. Then, $\forall \epsilon > 0, \forall n \geq 2N, |x_n - l| < \epsilon$ and $|y_n - l| < \epsilon$. Since $(z_n)$ is composed by alternating terms of $(x_n)$ and $(y_n)$, $\forall \epsilon > 0, \forall n \geq N, |z_n - l| < \epsilon$, which means $(z_n)$ converges to $l$. 

        \textit{Backwards: }

        If $(z_n)$ converges to $l$, then $\forall \epsilon > 0, \exists N \in \mathbf{N}$ s.t. $\forall n \geq N, |z_n - l| < \epsilon$. This includes $|z_{2n} - l|$ and $|z_{2n-1} - l|$. Since $(z_n)$ is composed by alternating terms of $(x_n)$ and $(y_n)$, this implies that $\forall \epsilon > 0, \exists N \in \mathbf{N}$ s.t. $\forall n \geq N, |y_n - l| < \epsilon$ and $\forall \epsilon > 0, \exists N \in \mathbf{N}$ s.t. $\forall n \geq N, |x_n - l| < \epsilon$. Thus, $(x_n)$ converges, $(y_n)$ converges, and $\text{lim }x_n = \text{lim }y_n$.
        
    \item \textbf{Exercise 2.3.11}
        \begin{enumerate}
            \item We have that $(x_n) \rightarrow x$, and we want to show that $(y_n) \rightarrow x$ as well. 

            Since $(x_n) \rightarrow x$, we have that $\forall \epsilon > 0, \exists N \in \mathbf{N}$ s.t. $\forall n \geq N, |x_n - x| < \epsilon$. 

            Consider \begin{align*}
                |y_n - x| &= \left|\frac{x_1 + x_2 + \dots + x_n}{n} - x\right| \\
                &= \left|\frac{x_1 - x}{n} + \frac{x_2 - x}{n} + \dots + \frac{x_n - x}{n}\right| \\
                &\leq \left|\frac{x_1 - x}{n}\right| + \left|\frac{x_2 - x}{n}\right| + \dots + \left|\frac{x_n - x}{n}\right|
            \end{align*}

            We know that $\forall \epsilon > 0, \exists N_1 \in \mathbf{N}$ s.t. $\forall n \geq N_1, \frac{|x_1 - x|}{n} < \frac{\epsilon}{n}$, and similarly for $x_2, x_3, \dots, x_n$. Choose $N = \text{max}(N_1, N_2, \dots N_n)$.

            Then, $\forall \epsilon > 0, \forall n \geq N$, we have $|y_n - x| \leq \left|\frac{x_1 - x}{n}\right| + \left|\frac{x_2 - x}{n}\right| + \dots + \left|\frac{x_n - x}{n}\right| < n \cdot \frac{\epsilon}{n} = \epsilon$, and so $(y_n) \rightarrow x$

            \item Let $x_n = (-1)^n$. We have already seen that $(x_n)$ does not converge, since it contains subsequences that converge to different limits. 
            
            Then $|y_n| = 0$ when $n$ is even and $\frac{1}{n}$ when $n$ is odd. If $n$ is even, then there are the same amount of $1$ and $-1$ in the numerator, so they cancel to 0, and if $n$ is odd, then there is one additional $-1$, and we still divide by $n$. So, $|y_n| \leq |\frac{1}{n}|$

            Then, $\forall \epsilon > 0, \exists N \in \mathbf{N}$ s.t. $\forall n \geq N, |y_n| \leq |\frac{1}{n}| < \epsilon$ due to the Archimedean Property. Thus, $(y_n)$ converges despite $(x_n)$ diverging. 
        \end{enumerate}
    \item \textbf{Exercise 2.5.2}
        \begin{enumerate}
            \item We will prove this by contrapositive, showing that if $(x_n)$ diverges, then it contains a divergent subsequence. 
            
            Let $y_n = x_{n+1}$. We want to show that $\forall x, \exists \epsilon > 0$ s.t. $\forall N \in \mathbf{N}, \exists n \geq N$ s.t. $|y_n - x| \geq \epsilon$. Since $y_n = x_{n+1}$, and since $(x_n)$ is divergent, we have that $\forall x, \exists \epsilon$ s.t. $\forall N \in \mathbf{N}, \exists n \geq N$ s.t. $|x_n - x| \geq \epsilon$, but this statement implies the divergence of $(y_n)$, and so if $(x_n)$ is divergent, it contains a divergent subsequence. 

            \item We will prove this also by contrapositive, showing that if $(x_n)$ is convergent, then all its subsequences $(a_{n_k})$ converge. 

            Since $(x_n)$ converges to $x$, $\forall \epsilon > 0, \exists N \in \mathbf{N}$ s.t. $\forall n \geq N, |x_n - x| < \epsilon$. However, since $n_k \geq k \forall k$, this implies that $|x_{n_k} - x| < \epsilon$, and so every subsequence of a convergent series converges. 

            \item Since $(x_n)$ is bounded, the sequence $(\text{sup}(x_n))$ converges by the monotone convergence theorem. Similarly, $(\text{inf}(x_n))$ converges, but since $(x_n)$ diverges, these limits must be different, and so we have found two subsequences that converge to different limits. 

            \item If $(x_{n_k}) \rightarrow x$, then $(x_{n_k})$ is bounded, and so $|x_{n_k}| \leq M$ for some $M > 0$. 

            Assume $(x_n)$ is increasing. Since it is monotone, $\text{min}(x_n) = x_1$, and since $(x_{n_k})$ is a subsequence, $x_1 \leq x_n \leq x_{n_k} \leq M$. Let $N = \text{max}(M, |x_1|)$, then $(x_n)$ is bounded by $N$.

            Similarly, assume $(x_n)$ is decreasing. Since it is monotone, $\text{max}(x_n) = x_1$, and since $(x_{n_k})$ is a subsequence, $-M \leq x_{n_k} \leq x_n \leq x_1$. Let $N = \text{max}(M, |x_1|)$, then $(x_n)$ is bounded by $N$.

            So, if $(x_n)$ is monotone and contains a convergent subsequence, $(x_n)$ is bounded, and since $(x_n)$ is montone and bounded, it converges. 
        \end{enumerate}
        
    \item \textbf{Exercise 2.5.9}

        By definition of the supremum, $\exists x \in S \text{ s.t. } s - \frac{1}{k} < x$, and by the definition of $S$, this means infinitely many terms $a_n$ satisfy $x < a_n$. Also by definition  of the supremum, $s + \frac{1}{k} \notin S$, so finitely many terms $a_n$ satisfy $a_n > s + \frac{1}{k}$, which is equivalent to saying infinitely many terms $a_n$ satisfy $a_n \leq s + \frac{1}{k}$.

        Combining these two inequalities, we get that there are infinitely many $a_n$ satisfying $s - \frac{1}{k} < a_n \leq s + \frac{1}{k}$.

        Definite the subsequence $a_{n_k}$ by choosing $a_{n_1}$ satisfying the inequality, and $a_{n_k}$ satisfying both the inequality and $n_k > n_{k-1}$. This is a valid step since we have infinitely many terms to choose from. 

        Thus, $\forall \epsilon > 0,$ choose $K \in \mathbf{N}, K > \frac{1}{\epsilon}$. Then, $\forall k \geq K, s - \epsilon < s - \frac{1}{k} < a_{n_k} \leq s + \frac{1}{k} < s + \epsilon$, and so $|a_{n_k} - s| < \epsilon$, so there is a subsequence that converges to the supremum of $S$.
    \item \textbf{Exercise 2.6.3}
        \begin{enumerate}
            \item Since $(x_n)$ is a Cauchy sequence, we have that $\forall \epsilon > 0, \exists N_1 \in \mathbf{N}$ s.t. $\forall n, m \geq N, |x_n - x_m| < \frac{\epsilon}{2}$, and similarly, $\forall \epsilon > 0, \exists N_2 \in \mathbf{N}$ s.t. $\forall n, m \geq N, |y_n - y_m| < \frac{\epsilon}{2}$.

            $\forall \epsilon > 0$, choose $N = \text{max}(N_1, N_2)$. Then, $\forall n, m \geq N,$ 
            \begin{align*}
                |(x_n + y_n) - (x_m + y_m)| &= |x_n - x_m + y_n + y_m| \\
                &\leq |x_n - x_m| + |y_n + y_m| \\
                &< \frac{\epsilon}{2} + \frac{\epsilon}{2} \\
                &= \epsilon
            \end{align*} 
            and so $(x_n + y_n)$ is Cauchy. 

            \item 
            Since $(x_n)$ and $(y_n)$ are Cauchy, they are bounded by $M_x$ and $M_y$ respectively. As well, we have that $\forall \epsilon > 0, \exists N_1 \in \mathbf{N}$ s.t. $\forall n, m \geq N, |x_n - x_m| < \frac{\epsilon}{2M_y}$, and similarly, $\forall \epsilon > 0, \exists N_2 \in \mathbf{N}$ s.t. $\forall n, m \geq N, |y_n - y_m| < \frac{\epsilon}{2M_x}$.

            $\forall \epsilon > 0$, choose $N = \text{max}(N_1, N_2)$. Then, $\forall n, m \geq N,$ 
            \begin{align*}
                |x_ny_n - x_my_m| &= |x_ny_n - x_ny_m + x_ny_m - x_my_m| \\
                &\leq |x_ny_n - x_ny_m| + |x_ny_m - x_my_m| \\
                &= |x_n||y_n - y_m| + |y_m||x_n - x_m| \\
                &\leq M_x|y_n - y_m| + M_y|x_n - x_m| \\
                &< \frac{\epsilon}{2} + \frac{\epsilon}{2} \\
                &= \epsilon
            \end{align*}
            and so $(x_ny_n)$ is Cauchy. 
        \end{enumerate}
\end{enumerate}

\end{document}
